\chapter{Aave}

\section{Overview}
Aave is a decentralized lending and borrowing protocol built on the Ethereum blockchain. It enables users to lend and borrow a wide range of cryptocurrencies directly, without the need for intermediaries such as banks. Aave operates through liquidity pools and smart contracts, providing a secure, transparent, and efficient platform for decentralized finance (DeFi) activities.
\\[5mm]
Users can deposit their cryptocurrency assets into Aave's liquidity pools and earn interest on their deposits. These funds contribute to the available liquidity for borrowers. On the other hand, borrowers can use their deposited assets as collateral to borrow other assets from the pool. The amount they can borrow is determined by the value of their collateral and specific borrowing parameters set by the protocol.

\section{How it works}
\subsection{Liquidity Pools}
The fundamental mechanism that enables Aave's functionality of lending and borrowing is liquidity pools. Users deposit their cryptocurrency assets into Aave's liquidity pools. These assets serve as collateral and contribute to the overall liquidity of the protocol. The deposited assets in the liquidity pools create reserves of available liquidity. These reserves are utilized to fulfill borrowing demands from other users within the Aave ecosystem.

\subsection{Lending and Borrowing}
Lending works by lenders depositing their cryptocurrency assets into Aave's liquidity pools. These assets act as collateral and are represented by interest-bearing tokens called aTokens. The aTokens represent the user's share of the deposited assets. Interest is earnt immediately and is accrued in real-time and reflected in an increase in the quantity of aTokens held by the depositor.
\\[5mm]
Borrowing works by borrowers using their deposited assets as collateral to borrow other assets from the liquidity pools. The value of the collateral determines the borrowing capacity. The borrowing capacity is calculated by paramters set by Aave, one of which is the maximum loan-to-value (LTV) ratio. Once the borrower requests a loan, if the borrower's collateral meets the necessary requirements, they can proceed with the loan and the borrowed funds are transferred to the borrower's wallet. In addition to this, borrowers pay interest on the borrowed amount based on the prevailing interest rates. Aave offers both variable and stable interest rates for borrowers. Variable rates fluctuate based on market dynamics, while stable rates remain fixed. This flexibility allows users to choose the borrowing option that best suits their needs. The interest on a loan is accrued in real-time, second by second, and the borrower decides when to repay it, as long as the loan is safe from liquidation. Liquidation is what happens if the value of a borrower's collateral falls below the liquidation threshold due to market volatility or other factors, the collateral may be liquidated. Liquidators can purchase the collateral at a discounted price to ensure the solvency of the liquidity pool.
